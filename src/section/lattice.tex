\begin{frame}
	\frametitle{\secname}

	\begin{definition}[Poset]
		Sea $\mathcal{L}\neq\emptyset$ un conjunto.
		Una \alert{relación de orden parcial} $\leq$ en $\mathcal{L}$ es
		una relación binaria en $\mathcal{L}$ que cumple la
		\begin{description}
			\item[reflexiva]

				\begin{math}
					\forall a\in\mathcal{L}:
					a\leq a.
				\end{math}

			\item[antisimétrica]

				\begin{math}
					\forall a,b\in\mathcal{L}:
					a\leq b
					\text{ y }
					b\leq a\implies
					a=b.
				\end{math}

			\item[transitiva]

				\begin{math}
					\forall a,b,c\in\mathcal{L}:
					a\leq b
					\text{ y }
					b\leq c\implies
					a\leq c.
				\end{math}
		\end{description}
		Si $\leq$ es una relación de orden parcial en $\mathcal{L}$,
		entonces $\left(\mathcal{L},\leq\right)$ es un
		\alert{conjunto parcialemente ordenado}.
	\end{definition}
	\begin{definition}[Látice]
		Un conjunto parcialmente
		ordenado $\left(\mathcal{L},\leq\right)$ es \alert{látice} sii
		$\forall a,b\in\mathcal{L}$ tiene un supremo, $a\wedge b$ y tiene
		un ínfimo $a\vee b$.
	\end{definition}

	\begin{definition}[Látice vectorial o Espacio de Riesz]
		Un \alert{látice vectorial} $V$ es un $\mathds{R}$-espacio
		vectorial, que tiene un orden en cual este es un látice, con las
		propiedades
		\begin{equation*}
			a\leq b\implies
			x+a\leq x+b,\quad
			\lambda\in\left[0,\infty\right), a\leq b\implies
			\lambda a\leq\lambda b\vee\wedge.
		\end{equation*}
	\end{definition}

	% \begin{proposition}
	% 	Si $S\subset C_{\mathds{R}}\left(X\right)$.
	% \end{proposition}
\end{frame}