% TODO: L^{2} is a Hilbert space.

% \begin{frame}
% 	El análisis de Fourier con la ecuación de la onda
% 	\begin{math}
% 		\diff[2]{y}{t}=
% 		c{2}\diff[2]{y}{x}
% 	\end{math}.

% 	Sea $\left(X,\left\langle,\right\rangle\right)$ un espacio
% 	pre-Hilbert el espacio de las funciones
% 	$f\colon\left[0,2\pi\right]\to\mathds{R}$
% \begin{math}
% 	\left(\alpha f+g\right)\left(x\right)=
% 	\alpha f\left(x\right)+g\left(x\right)
% \end{math}
% \begin{math}
% 	\left\langle f,g\right\rangle=
% 	\int_{0}^{2\pi}
% 	f\left(x\right)
% 	g\left(x\right)
% 	\dl x
% \end{math}
% 	\begin{block}{Serie de Fourier generalizada}
% 		Sea
% 		\begin{math}
% 			\left\{
% 			\varphi_{n}:\left[0,2\pi\right]\to\mathds{R}
% 			\right\}_{n\in\mathds{N}}\subset
% 			L\left(\left[0,2\pi\right]\right)
% 		\end{math}
% 		una sucesión ortonormal dada por
% 	\end{block}
% \end{frame}

% Si $f\in L\left(\left[0,2\pi\right]\right)$, entonces
% $a_{n},b_{n}<\infty$.
% \textcolor{DarkBlue}{a_{k}}

% \begin{definition}[Función superior]
% 	Una función $f\colon I\to\mathds{R}$ es \alert{superior} en $I$
% sii
% 	$\exists\left\{s_{n}\right\}_{n\in\mathds{N}}$ tal que
% 	\begin{multicols}{2}
% 		\begin{itemize}
% 			\item $s_{n}\to f$ casi en todas partes de $I$.
% 		\end{itemize}
% 	\end{multicols}
% \end{definition}

% \begin{definition}[Convergencia de una sucesión]
% 	Una sucesión
% 	\begin{math}
% 		{\left\{z_{n}\right\}}_{n\in\mathds{N}}\subset
% 		\mathds{C}
% 	\end{math}
% 	es \alert{convergente} sii $\exists z\in\mathds{C}$ tal que
% 	\begin{math}
% 		\forall\varepsilon>0:
% 		\exists N_{\varepsilon}\in\mathds{N}
% 	\end{math}
% 	tal que
% 	\begin{math}
% 		\forall n\geq N_{\varepsilon}:
% 		\left|z_{n}-z\right|<\varepsilon
% 	\end{math}.

% La \alert{sucesión de sumas parciales} se define como
% \begin{math}
% 	s_{n}\coloneqq\sum\limits_{k=1}^{n}z_{k}
% \end{math}.
% La serie
% \begin{math}
% 	\sum\limits_{k\in\mathds{N}}z_{k}
% \end{math}
% es \alert{convergente} al número $s\in\mathds{C}$ sii
% la sucesión
% \begin{math}
% 	{\left\{s_{n}\right\}}_{n\in\mathds{N}}\subset
% 	\mathds{C}
% \end{math}
% converge a $s$.
% \end{definition}

% https://proofwiki.org/wiki/Ces%C3%A0ro_Mean
% \begin{definition}[Sumación de Cesàro]
% 	Sean
% 	\begin{math}
% 		{\left\{z_{n}\right\}}_{n\in\mathds{N}}\subset
% 		\mathds{C}
% 	\end{math}
% 	y $\left\{s_{n}\right\}_{n\in\mathds{N}}$ su sucesión de sumas
% 	parciales.
% 	La \alert{sucesión de las medias aritméticas} se define como
% 	\begin{math}
% 		\sigma_{n}\coloneqq
% 		\dfrac{1}{n}\sum\limits_{k=1}^{n}s_{k}
% 	\end{math}.
% 	Si ${\left\{\sigma_{n}\right\}}_{n\in\mathds{N}}$ es convergente,
% 	entonces la serie
% 	\begin{math}
% 		\sum\limits_{n\in\mathds{N}}z_{n}
% 	\end{math}
% 	es \alert{sumable de Cesàro}.
% \end{definition}

% \begin{definition}[Convergencia uniforme]
% 	Sea $D\subset\mathds{C}$.
% 	\begin{math}
% 		\forall n\in\mathds{N}:f_{n}\colon D\to\mathds{C}
% 	\end{math}
% 	una función.
% 	La sucesión de funciones
% 	\begin{math}
% 		{\left\{f_{n}\right\}}_{n\in\mathds{N}}
% 	\end{math}
% 	\alert{converge uniformemente} a la función
% 	$f\colon D\to\mathds{C}$ sii
% 	\begin{math}
% 		\forall\varepsilon>0:\exists N\in\mathds{N}
% 	\end{math}
% 	tal que
% 	\begin{math}
% 		\forall n\geq N:
% 		{\left\|f_{n}-f\right\|}_{\infty}<
% 		\varepsilon.
% 	\end{math}
% \end{definition}

% TODO: .
% \frac{1}{n\pi}
% \frac{
% 	\sin^{2}\left(n\frac{\xi}{2}\right)
% }{
% 	\alert{\sin^{2}\left(\frac{\xi}{2}\right)}
% }
% \frac{1}{n\pi}
% \frac{
% 	\int\limits_{\delta}^{\pi}
% 	\left|
% 	g_{\theta}\left(\xi\right)
% 	\right|
% 	\dl\xi
% }{\alert{\sin^{2}\left(\frac{\delta}{2}\right)}}
% \leq
% \frac{
% 	\int\limits_{0}^{\pi}
% 	\left|
% 	g_{\theta}\left(\xi\right)
% 	\right|
% 	\dl\xi
% }{n\pi\sin^{2}\left(\frac{\delta}{2}\right)}.
% \shortintertext{
% 	Sea $N\in\mathds{N}$ tal que
% 	$
% 		\dfrac{
% 			\int\limits_{0}^{\pi}
% 			\left|
% 			g_{\theta}\left(\xi\right)
% 			\right|
% 			\dl\xi
% 		}{
% 			N\pi\sin^{2}
% 			\left(\frac{\delta}{2}\right)
% 		}<
% 		\dfrac{\varepsilon}{2}
% 	$.
% 	Entonces, $\forall n\geq N$:
% 	$
% 		\displaystyle
% 		\left|
% 		\sigma_{n}\left(\theta\right)-
% 		s\left(\theta\right)
% 		\right|\leq
% 		\left|
% 		\frac{1}{n\pi}
% 		\int\limits_{0}^{\pi}
% 		g_{\theta}\left(\xi\right)
% 		\frac{
% 			\sin^{2}\left(n\frac{\xi}{2}\right)
% 		}{
% 			\sin^{2}\left(\frac{\xi}{2}\right)
% 		}
% 		\dl\xi
% 		\right|
% 		<\varepsilon.
% 	$
% }

% \begin{frame}
% 	\begin{theorem}
% 		Sean $f\in L\left(\left(0,2\pi\right)\right)$ $2\pi$-periódica y
% 		$\left\{s_{n}\left(f\right)\right\}$ la $n$-ésima suma parcial de
% 		la serie de Fourier de $f$, y $\sigma_{n}\left(f\right)$ la
% 		sucesión de sumación Cesàro de la sucesión $s_{n}\left(f\right)$.
% 		Entonces, la sucesión $s_{n}\left(f\right)$ converge.
% 	\end{theorem}
% \end{frame}

% \begin{frame}
% 	\begin{proof}[\proofname\ (Cont.)]
% 		Sean $n\in\mathds{N}$ y $\xi\in\left[0,\delta\right]$.
% 		\begin{align*}
% 			\left|
% 			\sigma_{n}\left(\theta\right)-
% 			s\left(\theta\right)
% 			\right| & =
% 			\left|
% 			\frac{1}{n\pi}
% 			\int\limits_{0}^{\delta}
% 			\left(
% 			\alert{
% 				\frac{
% 					f\left(\theta+\xi\right)+
% 					f\left(\theta-\xi\right)
% 				}{2}-s\left(\theta\right)
% 			}
% 			\right)
% 			\frac{
% 				\sin^{2}\left(n\frac{\xi}{2}\right)
% 			}{
% 				\sin^{2}\left(\frac{\xi}{2}\right)
% 			}
% 			\dl\xi
% 			\right|=
% 			\left|
% 			\frac{1}{n\pi}
% 			\int\limits_{0}^{\delta}
% 			\alert{g_{\theta}\left(\xi\right)}
% 			\frac{
% 				\sin^{2}\left(n\frac{\xi}{2}\right)
% 			}{
% 				\sin^{2}\left(\frac{\xi}{2}\right)
% 			}
% 			\dl\xi
% 			\right|        \\
% 			        & \leq
% 			\frac{1}{n\pi}
% 			\left|
% 			\int\limits_{0}^{\delta}
% 			\alert{g_{\theta}\left(\xi\right)}
% 			\frac{
% 				\sin^{2}\left(n\frac{\xi}{2}\right)
% 			}{
% 				\sin^{2}\left(\frac{\xi}{2}\right)
% 			}
% 			\dl\xi
% 			\right|\leq
% 			\frac{1}{n\pi}
% 			\left|
% 			\int\limits_{0}^{\delta}
% 			\alert{\frac{\varepsilon}{2}}
% 			\frac{
% 				\sin^{2}\left(n\frac{\xi}{2}\right)
% 			}{
% 				\sin^{2}\left(\frac{\xi}{2}\right)
% 			}
% 			\dl\xi
% 			\right|=
% 			\frac{\varepsilon}{2}
% 			\frac{2}{n\pi}
% 			\alert{
% 				\int\limits_{0}^{\pi}
% 				\frac{1}{2n}
% 				\frac{
% 					\sin^{2}\left(n\frac{\xi}{2}\right)
% 				}{
% 					\sin^{2}\left(\frac{\xi}{2}\right)
% 				}
% 				\dl\xi
% 			}
% 			=\frac{\varepsilon}{2}\cdot\alert{1}.
% 		\end{align*}

% 		Sea $\theta\in\left[0,2\pi\right]$.
% 		Como $f$ es continua en $\left[0,2\pi\right]$, entonces
% 		$f$ es uniformemente continua en $\left[0,2\pi\right]$.

% 		Es decir,
% 		\begin{math}
% 			\forall\varepsilon>0
% 			\exists\delta>0
% 		\end{math}
% 		tal que
% 		\begin{math}
% 			\forall\theta_{1},\theta_{2}\in\left[0,2\pi\right]:
% 			\left|
% 			\theta_{1}-\theta_{2}
% 			\right|<\delta\implies
% 			\left|
% 			f\left(\theta_{1}\right)-f\left(\theta_{2}\right)
% 			\right|<\varepsilon
% 		\end{math}.

% 		\begin{align*}
% 			\left|
% 			\sigma_{n}\left(\theta\right)-
% 			s\left(\theta\right)
% 			\right|\leq
% 			\frac{2}{\pi}
% 			\int\limits_{0}^{\pi}
% 			\left|g_{\theta}\left(\xi\right)\right|
% 			\left|K_{n}\left(\xi\right)\right|
% 			\dl\xi
% 			=
% 			\frac{2}{\pi}
% 			\int\limits_{0}^{\delta}
% 			\left|g_{\theta}\left(\xi\right)\right|
% 			K_{n}\left(\xi\right)
% 			\dl\xi
% 			+
% 			\frac{2}{\pi}
% 			\int\limits_{\delta}^{\pi}
% 			\left|g_{\theta}\left(\xi\right)\right|
% 			K_{n}\left(\xi\right)
% 			\dl\xi \\
% 			=
% 			\frac{2}{\pi}
% 			\int\limits_{0}^{\delta}
% 			\left|g_{\theta}\left(\xi\right)\right|
% 			K_{n}\left(\xi\right)
% 			\dl\xi
% 			+
% 			\frac{2}{\pi}
% 			\int\limits_{\delta}^{\pi}
% 			\left|g_{\theta}\left(\xi\right)\right|
% 			K_{n}\left(\xi\right)
% 			\dl\xi \\
% 		\end{align*}

% 		Pero,
% 		\begin{math}
% 			\forall\xi\in\left[0,2\pi\right]:
% 			\exists\delta_{\xi}
% 		\end{math}
% 		tal que
% 		\begin{math}
% 			\left|
% 			g_{\theta}\left(\xi\right)
% 			\right|<
% 			\dfrac{\varepsilon}{2}
% 		\end{math}.

% 		Si $f$ es continua en $\left[0,2\pi\right]$, entonces
% 		$f$ es acotada en $\mathds{R}$, es decir, $\exists M>0$ tal que
% 		$\forall\theta,\xi\in\mathds{R}$:
% 		\begin{math}
% 			\left|g_{\theta}\left(\xi\right)\right|\leq M
% 		\end{math}.

% 		\begin{equation*}
% 			\left|
% 			\sigma_{n}\left(\theta\right)-
% 			s\left(\theta\right)
% 			\right|
% 			\leq
% 			\frac{1}{2\pi}
% 			\int_{0}^{\pi}
% 			\left|
% 			\right|
% 			K_{n}\left(\xi\right)
% 			\dl\xi
% 			<\varepsilon
% 		\end{equation*}
% 	\end{proof}
% \end{frame}

\begin{frame}
	\begin{equation*}
		L^{2}\left(\mathds{R}\right)=
		\left\{
		f\colon\mathds{R}\to\mathds{C}\mid
		\int_{\mathds{R}}\left|f\right|\dl\mu<\infty
		\right\}.
	\end{equation*}

	\begin{theorem}
		$L^{2}\left[0,1\right]\subset L^{1}\left[0,1\right]$.
	\end{theorem}

	\begin{proof}
		.
	\end{proof}

	\begin{theorem}
		The space $L^{2}\left[0,1\right]$ is meager in $L^{1}\left[0,1\right]$.
	\end{theorem}

	\begin{proof}
		.
	\end{proof}
\end{frame}
