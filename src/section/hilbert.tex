\begin{frame}

	\begin{theorem}[Parallelogram law]
		Let $\left(V,\left\|\cdot\right\|\right)$ be a normed
		$\mathds{C}$-vector space.
		Then,
		\begin{math}
			\forall x,y\in V:
			{\left\|x+y\right\|}^{2}+
			{\left\|x-y\right\|}^{2}
			=
			2\left(
			{\left\|x\right\|}^{2}
			+
			{\left\|y\right\|}^{2}
			\right)%.\tag{\text{\alert{Ley del paralelogramo}}}
		\end{math}.
	\end{theorem}

	% https://matthewhr.files.wordpress.com/2012/09/jordan-von-neumann-theorem.pdf
	\begin{theorem}[Jordan-von Neumann theorem~\cite{Jordan1935}]
		Let $\left(V,\left\|\cdot\right\|\right)$ be a normed
		$\mathds{C}$-vector space.
		$\left\|\cdot\right\|$ is induced by an inner product iff
		$\left\|\cdot\right\|$ holds the \alert{parallelogram law}.
	\end{theorem}

	\begin{theorem}
		Let $1\leq p<\infty$.
		The $L^{p}$-norm only holds the parallelogram law for $p=2$.
	\end{theorem}

	\begin{proof}
		Let $\left(X,\mathcal{F},\mu\right)$ be a measure space.
		Then, $\forall E\in\mathcal{F}$:
		\begin{equation*}
			{\left\|
			{\chi}_{E}
			\right\|}_{p}
				=
				{\left(
					\int_{X}
					{\left|\chi_{E}\right|}^{p}
					\dl\mu
					\right)}^{\frac{1}{p}}
				=
				{\left(
					\int_{E}
					{\chi_{E}}^{p}
					\dl\mu
					\right)}^{\frac{1}{p}}
			+
			{\left(
			\int_{E^{C}}
			{\chi_{E}}^{p}
			\dl\mu
			\right)}^{\frac{1}{p}}
				=
				{\left(
					\int_{E}
					1\,
					\dl\mu
					\right)}^{\frac{1}{p}}
			+
			{\left(
			\int_{E^{C}}
			0\,
			\dl\mu
			\right)}^{\frac{1}{p}}
				=
				{\mu\left(E\right)}^{\frac{1}{p}}+
			0
			=
			{\mu\left(E\right)}^{\frac{1}{p}}.
		\end{equation*}

		If $A,B\in\mathcal{F}$ such that $A\cap B=\emptyset$,
		$0<\mu\left(A\right)<\infty$ y $0<\mu\left(B\right)<\infty$,
		then
		\begin{math}
			\chi_{A}+
			\chi_{B}=
			\left|
			\chi_{A}-
			\chi_{B}
			\right|=
			\chi_{A\uplus B}
		\end{math}
		.
		\begin{align*}
			{\left\|\chi_{A}+\chi_{B}\right\|}^{2}_{p}+
			{\left\|\chi_{A}-\chi_{B}\right\|}^{2}_{p}
			 & =
			\left(
			\int_{X}
			{\left|
				\chi_{A}+
				\chi_{B}
				\right|}
			^{p}
			\dl\mu
			\right)^{\frac{2}{p}}
			+
			\left(
			\int_{X}
			{\left|
				\chi_{A}-
				\chi_{B}
				\right|}
			^{p}
			\dl\mu
			\right)^{\frac{2}{p}}
			=
			2\left(
			\int_{X}
			{\left|
				\chi_{A\uplus B}
				\right|}
			^{p}
			\dl\mu
			\right)^{\frac{2}{p}}
			=
			2
				{
					\left(
					\mu
					\left(
					A\uplus B
					\right)
					\right)
					^{\frac{2}{p}}
				}.
			\\
			\alert{2}\left(
			{\left\|\chi_{A}\right\|}^{2}_{p}
			+
			{\left\|\chi_{B}\right\|}^{2}_{p}
			\right)
			 & =
			\alert{2}
			\left(
			{
				\left(
				\mu\left(A\right)
				\right)
			}
			^{\frac{2}{p}}
			+
			{
				\left(
				\mu\left(B\right)
				\right)
			}
			^{\frac{2}{p}}
			\right).
		\end{align*}
		Hence, $L^{p}$-norm only holds the \alert{parallelogram law} for
		$p=2$.
	\end{proof}
\end{frame}

\begin{frame}
	\begin{definition}[Hilbert space]
		A \alert{Hilbert space}
		\begin{math}
			\left(
			H,
			\left\langle\cdot,\cdot\right\rangle
			\right)
		\end{math}
		is a pre-Hilbert space that is complete with respect to the norm
		\begin{math}
			\left\|\cdot\right\|=
			{\left\langle\cdot,\cdot\right\rangle}^{\frac{1}{2}}
		\end{math}.
	\end{definition}

	\begin{definition}[Orthonormal basis]
		Let
		\begin{math}
			\left(
			H,
			\left\langle\cdot,\cdot\right\rangle
			\right)
		\end{math}
		be a Hilbert space.
		An \alert{orthonormal basis} of $H$
		is a countable maximal orthonormal subset
		\begin{math}
			{\left\{e_{n}\right\}}_{n\in\mathds{N}}
		\end{math}
		of $H$.
	\end{definition}

	\begin{theorem}
		Let
		\begin{math}
			\left(
			H,
			\left\langle\cdot,\cdot\right\rangle
			\right)
		\end{math}
		be a Hilbert space and $\left\{e_{n}\right\}_{n\in\mathds{N}}$
		an orthonormal basis on $H$.
		Then, we have convergence of the
		\alert{Fourier-Bessel series}:
		\begin{equation*}
			\forall u\in H:
			\lim\limits_{n\to\infty}
			\sum_{k=1}^{n}
			\left\langle u,e_{k}\right\rangle
			e_{k}=
			\sum_{n=1}^{\infty}
			\left\langle u,e_{n}\right\rangle
			e_{n}
			=u.
		\end{equation*}
	\end{theorem}

	\begin{theorem}
		Let
		\begin{math}
			\left(
			H,
			\left\langle\cdot,\cdot\right\rangle
			\right)
		\end{math}
		be a Hilbert space.
		If $H$ has an orthonormal basis, then $H$ is \alert{separable}.
	\end{theorem}

	\begin{theorem}
		Let $1\leq p<\infty$.
		The $L^{p}\left(\mu\right)$ is a Hilbert space iff $p=2$.
	\end{theorem}

	\begin{proof}[\alert{A proof soon}]
		% \begin{itemize}
		% 	\item[$\Rightarrow$]

		% 		If $L^{p}\left(\mu\right)$ is Hilbert, then
		% 		$L^{p}\left(\mu\right)$ is reflexive.
		% 		I.e.
		% 		\begin{math}
		% 			{
		% 				\left(L^{p}\left(\mu\right)\right)
		% 			}^{\ast\ast}=
		% 			{
		% 			\left({\left(L^{p}\left(\mu\right)\right)}^{\ast}\right)
		% 			}^{\ast}=
		% 				{
		% 					\left(L^{q}\left(\mu\right)\right)
		% 				}^{\ast}=
		% 			L^{p}\left(\mu\right)
		% 		\end{math}.
		% 		But.

		% 	\item[$\Leftarrow$]

		% 		.
		% \end{itemize}
	\end{proof}
\end{frame}

% Sean $I\subset\mathds{R}$ un intervalo.
% \begin{math}
% 	{\left\{\varphi_{k}\right\}}_{\lambda\in\Lambda}\subset
% 	L^{2}\left(I\right)
% \end{math}.

% \begin{definition}
% 	Suponga que $\left(X,\mathcal{F},\mu\right)$ es un espacio de
% 	medida y $0<p\leq\infty$.
% \end{definition}