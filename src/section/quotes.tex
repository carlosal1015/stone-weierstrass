\begin{frame}
	\alt<3>{
		\begin{quotation}
			``Un matemático que no es también algo poeta nunca será un
			matemático completo.''
			\begin{flushright}
				-- Karl Theodor Wilhelm Weierstraß (1815 - 1897)
			\end{flushright}
		\end{quotation}

		\

		\begin{quotation}
			``La ciencia es razonamiento; el razonamiento es matemática; y,
			por lo tanto, la ciencia es matemática.''
			\begin{flushright}
				-- Marshall Harvey Stone (1903 - 1989)
			\end{flushright}
		\end{quotation}

		\

		\begin{quotation}
			Cuando escribo un artículo,
			``Tengo que volver a derivar por mí mismo las reglas de
			derivación y, a veces, incluso la ley conmutativa de la
			multiplicación.''
			\begin{flushright}
				-- Lipót Fejér (1880 - 1959)
			\end{flushright}
		\end{quotation}
	}{
		\begin{quotation}
			\glqq Ein Mathematiker, der nicht irgendwie ein Dichter ist,
			wird nie ein vollkommener Mathematiker sein.\grqq
			\begin{flushright}
				-- Karl Theodor Wilhelm Weierstraß (1815 - 1897)
			\end{flushright}
		\end{quotation}
		% https://de.wikipedia.org/wiki/Karl_Weierstra%C3%9F

		\

		\begin{quotation}
			``Science is reasoning; reasoning is mathematics; and,
			therefore, science is mathematics.''
			\begin{flushright}
				-- Marshall Harvey Stone (1903 - 1989)
			\end{flushright}
		\end{quotation}
		% https://en.wikipedia.org/wiki/Marshall_H._Stone

		\

		\begin{quotation}
			When I write a paper,
			``I have to rederive for myself the rules of differentiation
			and sometimes even the commutative law of multiplication.''
			\begin{flushright}
				-- Lipót Fejér (1880 - 1959)
			\end{flushright}
		\end{quotation}
	}
\end{frame}