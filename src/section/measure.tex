\begin{frame}
	\frametitle{\secname}

	\begin{definition}[$\sigma$-algebra]
		Let $X$ be a set.
		A \alert{$\sigma$-algebra} on $X$ is a family
		$\mathcal{F}\subset 2^{X}$ that holds

		\begin{columns}
			\begin{column}{0.48\textwidth}
				\begin{itemize}
					\item

					      \begin{math}
						      \forall A\in\mathcal{F}\implies
						      X\setminus A\in\mathcal{F}
					      \end{math}.
				\end{itemize}
			\end{column}
			\begin{column}{0.48\textwidth}
				\begin{itemize}
					\item

					      \begin{math}
						      \forall
						      \left\{A_{i}\right\}_{i\in\mathds{N}}\subset
						      \mathcal{F}\implies
						      \bigcup\limits_{i\in\mathds{N}}A_{i}\in\mathcal{F}
					      \end{math}.
				\end{itemize}
			\end{column}
		\end{columns}
	\end{definition}

	\begin{definition}[Measure]
		Let $X$ be a set and $\mathcal{F}$ a $\sigma$-algebra on $X$.
		A \alert{measure} on $\left(X,\mathcal{F}\right)$ is a function
		\begin{math}
			\mu\colon\mathcal{F}\to\left[0,\infty\right]
		\end{math}
		that holds

		\begin{columns}
			\begin{column}{0.18\textwidth}
				\begin{itemize}
					\item

					      \begin{math}
						      \mu\left(\emptyset\right)=
						      0.
					      \end{math}
				\end{itemize}
			\end{column}
			\begin{column}{0.78\textwidth}
				\begin{itemize}
					\item

					      \begin{math}
						      \forall
						      {\left\{A_{i}\right\}}_{i\in\mathds{N}}\subset
						      \mathcal{F}:
						      \forall i\neq j:
						      A_{i}\cap A_{j}=\emptyset\implies
						      \mu\left(
						      \bigcup\limits_{i\in\mathds{N}}A_{i}
						      \right)=
						      \sum\limits_{i\in\mathds{N}}\mu\left(A_{i}\right)
					      \end{math}.
				\end{itemize}
			\end{column}
		\end{columns}
	\end{definition}

	\begin{definition}[Outer measure]
		Let $X$ be a set.
		An \alert{outer measure} on $X$ is a function
		\begin{math}
			\mu^{\ast}\colon 2^{X}\to\left[0,\infty\right]
		\end{math}
		that holds
		\begin{columns}
			\begin{column}{0.13\textwidth}
				\begin{itemize}
					\item

					      \begin{math}
						      \mu^{\ast}\left(\emptyset\right)=
						      0.
					      \end{math}
				\end{itemize}

			\end{column}
			\begin{column}{0.43\textwidth}
				\begin{itemize}
					\item

					      \begin{math}
						      \forall A,B\in 2^{X}:
						      A\subset B\implies
						      \mu^{\ast}\left(A\right)\leq
						      \mu^{\ast}\left(B\right)
					      \end{math}.
				\end{itemize}
			\end{column}
			\begin{column}{0.48\textwidth}
				\begin{itemize}
					\item

					      \begin{math}
						      \forall
						      \left\{A_{i}\right\}_{i\in\mathds{N}}\subset
						      2^{X}\implies
						      \mu^{\ast}
						      \left(\bigcup\limits_{i\in\mathds{N}}A_{i}\right)=
						      \sum\limits_{i\in\mathds{N}}\mu^{\ast}
						      \left(A_{i}\right)
					      \end{math}.
				\end{itemize}
			\end{column}
		\end{columns}
	\end{definition}

	\note{
		.
	}
\end{frame}

\begin{frame}
	\frametitle{\secname}

	\begin{definition}[$\sigma$-algebra generated]
		Let $X$ be a set and $\mathcal{G}\subset 2^{X}$.
		The \alert{$\sigma$-algebra generated} by $\mathcal{G}$ is the
		smallest $\sigma$-algebra on $X$ which contains $\mathcal{G}$.
		\begin{equation*}
			\sigma\left(\mathcal{G}\right)\coloneqq
			\bigcap\limits_{
				\mathcal{A}\in\mathcal{F}\left(\mathcal{G}\right)
			}
			\mathcal{A},\quad
			\mathcal{F}\left(\mathcal{G}\right)=
			\left\{
			\mathcal{A}\subset 2^{X}\mid
			\mathcal{G}\subset\mathcal{A},
			\mathcal{A}\text{ is a $\sigma$-algebra on $X$}
			\right\}.
		\end{equation*}
	\end{definition}

	\begin{definition}[Borel $\sigma$-algebra]
		Let $\left(X,\mathcal{T}\right)$ be a topological space.
		The \alert{Borel $\sigma$-algebra} on $X$ is
		$\sigma\left(\mathcal{T}\right)$.
		The sets in $\sigma\left(\mathcal{T}\right)$ are
		\alert{Borel sets}.
	\end{definition}

	\begin{definition}[Lebesgue measure]
		The \alert{Lebesgue measure} is a measure
		on $\left(\mathds{R},\mathcal{B}\right)$, where $\mathcal{B}$ is
		the Borel $\sigma$-algebra of subsets of $\mathds{R}$, which
		assigns each Borel set its outer measure.
	\end{definition}

	\begin{definition}[Lebesgue space $\mathcal{L}^{1}\left(\mu\right)$]
		Let $\left(X,\mathcal{F},\mu\right)$ be a measure space.
		If $f\colon X\to\left[-\infty,\infty\right]$ is
		$\mathcal{F}$-measurable, then the
		\alert{$\mathcal{L}^{1}$-norm} of $f$ is
		\begin{equation*}
			{\left\|f\right\|}_{1}\coloneqq
			\int\left|f\right|\dl\mu.
		\end{equation*}
		The \alert{Lebesgue space} $\mathcal{L}^{1}\left(\mu\right)$
		is
		\begin{math}
			\mathcal{L}^{1}\left(\mu\right)\coloneqq
			\left\{
			f\colon X\to\mathds{R}\mid
			f\text{ is a function $\mathcal{F}$-measurable and }
			\left\|f\right\|_{1}<\infty
			\right\}.
		\end{math}
	\end{definition}

	\note{
		Discuss a little the Borel sets and the Lebesgue measure.
	}
\end{frame}

\begin{frame}
	\begin{definition}[${\left\|f\right\|}_{p}$, essential supremum]
		Let $\left(X,\mathcal{F},\mu\right)$ be a measure space and
		$0<p<\infty$.
		If $f\colon X\to\mathds{C}$ is $\mathcal{F}$-measurable, then
		the \alert{$p$-norm} of $f$ is
		\begin{equation*}
			{\left\|f\right\|}_{p}\coloneqq
			{\left(
				\int{\left|f\right|}^{p}\dl\mu
				\right)}^{\frac{1}{p}}.
		\end{equation*}
		Also, the \alert{essential supremum} of $f$ is
		\begin{math}
			{\left\|f\right\|}_{\infty}=
			\inf
			\left\{
			t>0:
			\mu\left(
			\left\{x\in X:\left|f\left(x\right)\right|>t\right\}
			\right)=0
			\right\}.
		\end{math}
	\end{definition}

	\begin{theorem}
		Let $\left(X,\mathcal{F},\mu\right)$ be a measure space and
		$0<p<\infty$.
		Then,
		\alert{$\mathcal{L}^{p}\left(\mu\right)$ is a vector space}
		and it is holds:
		\begin{columns}
			\begin{column}{0.48\textwidth}
				\begin{itemize}
					\item

					      \begin{math}
						      \forall f,g\in\mathcal{L}^{p}\left(\mu\right):
						      {\left\|f+g\right\|}^{p}_{p}\leq
						      2^{p}
						      \left(
						      {\left\|f\right\|}^{p}_{p}+
						      {\left\|g\right\|}^{p}_{p}
						      \right)
					      \end{math}.
				\end{itemize}
			\end{column}
			\begin{column}{0.48\textwidth}
				\begin{itemize}
					\item

					      \begin{math}
						      \forall f\in\mathcal{L}^{p}\left(\mu\right):
						      \forall\alpha\in\mathds{C}:
						      {\left\|\alpha f\right\|}_{p}=
						      \left|\alpha\right|
						      {\left\|f\right\|}_{p}
					      \end{math}.

				\end{itemize}
			\end{column}
		\end{columns}
	\end{theorem}

	\begin{proof}
		Let $f,g\in\mathcal{L}^{p}\left(\mu\right)$, $0<p<\infty$,
		$x\in X$ and $\alpha\in\mathds{C}$.
		\begin{itemize}
			\item

			      Then,
			      \begin{math}
				      {\left|f\left(x\right)+g\left(x\right)\right|}^{p}
				      \leq
				      {
					      \left(
					      \left|f\left(x\right)\right|+
					      \left|g\left(x\right)\right|
					      \right)
				      }^{p}
				      \leq
				      {\left(
					      2\max
					      \left\{
					      \left|f\left(x\right)\right|,
					      \left|g\left(x\right)\right|
					      \right\}
					      \right)
				      }^{p}
				      \leq
				      2^{p}
				      \left(
				      {\left|f\left(x\right)\right|}^{p}+
				      {\left|g\left(x\right)\right|}^{p}
				      \right)
			      \end{math}.

			      Integrating both sides of the inequality with respect to
			      $\mu$:
			      \begin{math}
				      {\left\|f+g\right\|}^{p}_{p}
				      \leq
				      2^{p}
				      \left(
				      {\left\|f\right\|}^{p}_{p}+{\left\|g\right\|}^{p}_{p}
				      \right)
			      \end{math}.

			      I.e. if ${\left\|f\right\|}_{p}<\infty$ and
			      ${\left\|g\right\|}_{p}<\infty$, then
			      ${\left\|f+g\right\|}_{p}<\infty$.

			\item

			      \begin{math}
				      \displaystyle
				      {\left\|\alpha f\right\|}_{p}=
				      {
				      \left(
				      \int
				      {\left|\alpha f\right|}^{p}\dl\mu
				      \right)
				      }^{\frac{1}{p}}
					      =
					      {
						      \left(
						      \int
						      {\left|\alpha\right|}^{p}
						      {\left|f\right|}^{p}\dl\mu
						      \right)
					      }^{\frac{1}{p}}
					      =
					      {\left|\alpha\right|}^{\frac{p}{p}}
					      {
						      \left(
						      \int
						      {\left|f\right|}^{p}\dl\mu
						      \right)
					      }^{\frac{1}{p}}
				      =
				      \left|\alpha\right|
				      {\left\|f\right\|}_{p}
			      \end{math}.
		\end{itemize}

		Since $0\in\mathcal{L}^{p}\left(\mu\right)$,
		\begin{math}
			\mathcal{L}^{p}\left(\mu\right)<
			\mathds{C}^{X}
		\end{math}
		is closed under addition and scalar multiplication.
		$\therefore$
		\alert{$\mathcal{L}^{p}\left(\mu\right)$ is a vector space}.
	\end{proof}
\end{frame}

\begin{frame}
	\begin{block}{Remark~\cite{Liou2013}}
		Let $\left(X,\mathcal{F},\mu\right)$ be a measure space.
		The function
		\begin{align*}
			\mathcal{L}^{2}\left(\mu\right) & \to\mathds{R} \\
			f                               & \mapsto
			{\left(
				\int_{X}{{\left|f\right|}}^{2}\dl\mu
				\right)}^{\frac{1}{2}}
		\end{align*}
		\alert{is not a norm} on $\mathcal{L}^{2}\left(\mu\right)$
		because $\exists f\in\mathcal{L}^{2}\left(\mu\right)$ non-zero
		such that
		\begin{math}
			\displaystyle
			\int_{X}{{\left|f\right|}}^{2}\dl\mu=
			0\in
			\mathds{R}
		\end{math}.
		% So we define the equivalence relation on
	\end{block}

	\begin{definition}[$\mathcal{Z}\left(\mu\right)$, $\widetilde{f}$]
		Let $\left(X,\mathcal{F},\mu\right)$ be a measure space and
		$0<p\leq\infty$.
		We define
		\begin{itemize}
			\item

			      \begin{math}
				      \alert{\mathcal{Z}\left(\mu\right)}\coloneqq
				      \left\{
				      f\colon X\to\mathds{C}\mid
				      f\text{ is a function $\mathcal{F}$-measurable and }
				      \mu
				      \left(
				      \left\{x\in X:f\left(x\right)\neq0\right\}
				      \right)=
				      0
				      \right\}
			      \end{math}.

			\item

			      \begin{math}
				      \forall f\in\mathcal{L}^{p}\left(\mu\right):
				      \alert{\widetilde{f}}=
				      \left\{
				      f+z:z\in\mathcal{Z}\left(\mu\right)
				      \right\}<
				      \mathcal{L}^{p}\left(\mu\right)
			      \end{math}.
		\end{itemize}
		Note that if $f,F\in\mathcal{L}^{p}\left(\mu\right)$, then
		$\widetilde{f}=\widetilde{F}$ iff
		\begin{math}
			\mu\left(
			\left\{
			x\in X:f\left(x\right)\neq F\left(x\right)
			\right\}
			\right)=0
		\end{math}.
	\end{definition}

	\begin{definition}[$L^{p}\left(\mu\right)$ space]
		Let $\mu$ is a measure and $0<p\leq\infty$.
		The set $L^{p}\left(\mu\right)$ are the
		\alert{equivalence classes of functions} on
		\begin{math}
			\mathcal{L}^{p}\left(\mu\right)
		\end{math},
		where two functions are equivalent iff they are equal almost
		everywhere.
		\begin{columns}
			\begin{column}{0.42\textwidth}
				\begin{itemize}
					\item

					      \begin{math}
						      \alert{
							      L^{p}\left(\mu\right)
						      }
						      \coloneqq
						      \left\{
						      \widetilde{f}:
						      f\in\mathcal{L}^{p}\left(\mu\right)
						      \right\}=
						      \mathcal{L}^{p}
						      \left(\mu\right)/\mathcal{Z}\left(\mu\right).
					      \end{math}
				\end{itemize}
			\end{column}
			\begin{column}{0.54\textwidth}
				\begin{itemize}
					\item

					      \begin{math}
						      \forall\widetilde{f},\widetilde{g}\in
						      L^{p}\left(\mu\right):
						      \forall\alpha\in\mathds{C}:
						      \widetilde{f}+\widetilde{g}\coloneqq
						      {\left(f+g\right)}^{\widetilde{}},\quad
						      \alpha\widetilde{f}\coloneqq
						      {\left(\alpha f\right)}^{\widetilde{}}.
					      \end{math}
				\end{itemize}
			\end{column}
		\end{columns}
	\end{definition}
\end{frame}

\begin{frame}
	\begin{definition}[${\left\|\cdot\right\|}_{p}$ on $L^{p}\left(\mu\right)$]
		Let $\mu$ be a measure and $0<p\leq\infty$.
		We define
		\begin{math}
			\forall f\in\mathcal{L}^{p}\left(\mu\right):
			{\left\|\widetilde{f}\right\|}_{p}=
				{\left\|f\right\|}_{p}
		\end{math}.

		Note that if $f,F\in\mathcal{L}^{p}\left(\mu\right)$
		and $\widetilde{f}=\widetilde{F}$, then
		\begin{math}
			{\left\|f\right\|}_{p}=
				{\left\|F\right\|}_{p}
		\end{math}.
	\end{definition}

	\begin{theorem}
		Let $\mu$ be a measure and $p\leq 1\leq\infty$.
		Then, $L^{p}\left(\mu\right)$ is a \alert{vector space} and
		${\left\|\cdot\right\|}_{p}$ is a norm on
		$L^{p}\left(\mu\right)$.
	\end{theorem}

	\begin{proof}[\alert{A proof soon}]
		Let
		\begin{math}
			\widetilde{f},\widetilde{g}\in
			L^{p}\left(\mu\right)
		\end{math}
		and $\alpha\in\mathds{C}$.

		% Since,
		% \begin{math}
		% 	\forall 1\leq q\leq p:
		% 	L^{p}\left(\mu\right)\subset
		% 	L^{q}\left(\mu\right)
		% \end{math}.
	\end{proof}
\end{frame}

\begin{frame}
	\begin{definition}[Convergent sequence]
		Let $\left(X,\left\|\cdot\right\|\right)$ be a normed
		$\mathds{C}$-vector space.
		A sequence
		\begin{math}
			\left\{f_{n}\right\}_{n\in\mathds{N}}\subset
			X
		\end{math}
		is a \alert{convergent sequence} iff $\exists f\in X$ such that
		\begin{math}
			\forall\varepsilon>0:
			\exists N\in\mathds{N}
		\end{math}
		such that
		\begin{math}
			\forall n\geq N:
			\left\|f-f_{n}\right\|<
			\varepsilon
		\end{math}.
	\end{definition}

	\begin{definition}[Cauchy sequence]
		Let $\left(X,\left\|\cdot\right\|\right)$ be a normed
		$\mathds{C}$-vector space.
		A sequence
		\begin{math}
			\left\{f_{n}\right\}_{n\in\mathds{N}}\subset
			X
		\end{math}
		is a \alert{Cauchy sequence} iff
		\begin{math}
			\forall\varepsilon>0:\exists N\in\mathds{N}
		\end{math}
		such that
		\begin{math}
			\forall m,n\geq N:
			\left\|f_{m}-f_{n}\right\|<\varepsilon
		\end{math}.
		$\left(X,\left\|\cdot\right\|\right)$ is \alert{complete} iff
		each Cauchy sequence in $X$ is convergent in $X$.
	\end{definition}

	\begin{theorem}
		Let $\left(X,\left\|\cdot\right\|\right)$ be a normed
		$\mathds{C}$-vector space and
		${\left\{f_{n}\right\}}_{n\in\mathds{N}}\subset X$ a sequence.
		If
		\begin{math}
			{\left\{f_{n}\right\}}_{n\in\mathds{N}}
		\end{math}
		is convergent in X, then
		\begin{math}
			{\left\{f_{n}\right\}}_{n\in\mathds{N}}
		\end{math}
		is Cauchy in $X$.
		Also, if
		\begin{math}
			{\left\{f_{n}\right\}}_{n\in\mathds{N}}
		\end{math}
		is Cauchy in $X$ and has a
		\alert{convergent subsequence} in $X$, then
		\begin{math}
			{\left\{f_{n}\right\}}_{n\in\mathds{N}}
		\end{math}
		converges in $X$.
	\end{theorem}

	\begin{proof}
		Let ${\left\{f_{n}\right\}}_{n\in\mathds{N}}\subset X$ a
		convergent sequence.
		I.e. $\exists f\in X$ such that
		\begin{math}
			\forall\varepsilon>0:
			\exists N\in\mathds{N}
		\end{math}
		such that
		\begin{math}
			\forall n\geq N:
			\left\|f-f_{n}\right\|<
			\frac{\varepsilon}{2}
		\end{math}.

		Hence,
		\begin{math}
			\forall\varepsilon>0:
			\exists N\in\mathds{N}
		\end{math}
		such that
		\begin{math}
			\forall m,n\geq N:
			\left\|f_{m}\alert{-f+f}-f_{n}\right\|\leq
			\left\|f_{m}-f\right\|+
			\left\|f-f_{n}\right\|<
			\frac{\varepsilon}{2}+
			\frac{\varepsilon}{2}=
			\varepsilon
		\end{math}.

		\

		Let ${\left\{f_{n}\right\}}_{n\in\mathds{N}}\subset X$
		a Cauchy sequence that has a convergent subsequence
		${\left\{f_{n_{k}}\right\}}_{k\in\mathds{N}}\subset X$.
		Since ${\left\{f_{n}\right\}}_{n\in\mathds{N}}$ is a Cauchy, i.e.
		\begin{math}
			\exists N\in\mathds{N}
		\end{math}
		such that
		\begin{math}
			\forall m,n\geq N:
			\left\|f_{m}-f_{n}\right\|<
			\frac{\varepsilon}{2}
		\end{math}.
		Also ${\left\{f_{n_{k}}\right\}}_{k\in\mathds{N}}$ is convergent,
		i.e. $\exists n_{k}>N$ such that
		\begin{math}
			\left\|f-f_{n_{k}}\right\|<
			\frac{\varepsilon}{2}
		\end{math}.

		Therefore,
		\begin{math}
			\forall\varepsilon>0:
			\exists N\in\mathds{N}
		\end{math}
		such that
		\begin{math}
			\forall n\geq N:
			\left\|f\alert{-f_{n_{k}}+f_{n_{k}}}-f_{n}\right\|\leq
			\left\|f-f_{n_{k}}\right\|+
			\left\|f_{n_{k}}-f_{n}\right\|<
			\frac{\varepsilon}{2}+
			\frac{\varepsilon}{2}=
			\varepsilon
		\end{math}.
	\end{proof}
\end{frame}

\begin{frame}
	\begin{theorem}[Riesz - Fischer theorem]
		Let $\left(X,\mathcal{F},\mu\right)$ be a measure space and
		$1\leq p\leq\infty$.
		Then, $L^{p}\left(\mu\right)$ is a Banach space.
	\end{theorem}

	\begin{proof}[\alert{A proof soon}]
		Let
		\begin{math}
			{\left\{\widetilde{f}_{n}\right\}}_{n\in\mathds{N}}\subset
			L^{p}\left(\mu\right)
		\end{math}
		a Cauchy sequence, i.e.
		\begin{math}
			\forall\varepsilon>0:
			\exists N\in\mathds{N}
		\end{math}
		such that
		\begin{math}
			\forall m,n\geq N:
			{\left\|\widetilde{f}_{m}-\widetilde{f}_{n}\right\|}_{p}<
			\frac{\varepsilon}{2}
		\end{math}.

		% \begin{math}
		% 	\displaystyle
		% 	{\left\|\widetilde{f}_{m}-\widetilde{f}_{n}\right\|}_{p}=
		% 	{\left\|\widetilde{f}_{m}\alert{-f+f}-\widetilde{f}_{n}\right\|}_{p}\leq
		% 	{\left\|\widetilde{f}_{m}-f\right\|}_{p}+
		% 	{\left\|f-\widetilde{f}_{n}\right\|}_{p}=
		% 		{\left(\int_{X}{\left|f_{m}-f\right|}^{p}\dl\mu\right)}^{\frac{1}{p}}+
		% 	{\left(\int_{X}{\left|f-f_{n}\right|}^{p}\dl\mu\right)}^{\frac{1}{p}}
		% \end{math}.

		% Hence, $\exists\widetilde{f}\in L^{p}\left(\mu\right)$ such that
		% \begin{math}
		% 	\forall\varepsilon>0:
		% 	\exists N\in\mathds{N}
		% \end{math}
		% such that
		% \begin{math}
		% 	\forall n\geq N:
		% \end{math}

		% \begin{equation*}
		% 	{\left\|\widetilde{f}-\widetilde{f}_{n}\right\|}_{p}=
		% 	{\left\|
		% 	\widetilde{f}-\widetilde{f}_{n_{k}}+
		% 	\widetilde{f}_{n_{k}}-\widetilde{f}_{n}
		% 	\right\|}_{p}\leq
		% 	{\left\|\widetilde{f}-\widetilde{f}_{n_{k}}\right\|}_{p}+
		% 	{\left\|\widetilde{f}_{n_{k}}-\widetilde{f}_{n}\right\|}_{p}
		% 	<
		% 	\frac{\varepsilon}{2}+
		% 	\frac{\varepsilon}{2}
		% 	<\varepsilon.
		% \end{equation*}
	\end{proof}

	% We say that $f_{n}\xrightarrow{L^{2}}f$ iff
	% \begin{math}
	% 	\left\|f-f_{n}\right\|\to0
	% \end{math}.
	% Suppose that $\left\{f_{n}\right\}$ is a Cauchy sequence
	% $\exists f\in L^{2}$ such that $f_{n}\xrightarrow{L^{2}}f$.
\end{frame}