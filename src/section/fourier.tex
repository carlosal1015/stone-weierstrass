\begin{frame}
	\frametitle{\secname}
	\begin{definition}[Fourier series of $f$ relative]
		Let $f\in L^{2}\left(I\right)$ and
		${\left\{\varphi_{k}\right\}}_{k\in\mathds{N}}$ an orthonormal
		sequence on $I$.
		The \alert{Fourier series of $f$ relative} of
		${\left\{\varphi_{k}\right\}}_{k\in\mathds{N}}$ is
		\begin{math}
			\displaystyle
			\sum_{k\in\mathds{N}}
			c_{k}\varphi_{k}\left(\theta\right),
		\end{math}
		where
		\begin{math}
			\forall k\in\mathds{N}:
			c_{k}\coloneqq
			\left\langle f,\varphi_{k}\right\rangle=
			\displaystyle\int_{I}
			f\left(\theta\right)\overline{\varphi_{k}\left(\theta\right)}
		\end{math}
		are the \alert{Fourier coefficients of $f$ relative} of
		${\left\{\varphi_{k}\right\}}_{k\in\mathds{N}}$.
	\end{definition}

	\begin{block}{Example}
		If $I=\left[0,2\pi\right]$ and two
		orthonormal sequences of trigonometric functions
		\begin{math}
			{\left\{\varphi_{k}\right\}}_{k\in\mathds{N}},
			{\left\{\phi_{k}\right\}}_{k\in\mathds{Z}}
		\end{math}:
		\begin{description}
			\item[real]

				\begin{math}
					\varphi_{0}\left(\theta\right)=
					\dfrac{1}{\sqrt{2\pi}},\quad
					\varphi_{2k-1}\left(\theta\right)=
					\dfrac{\cos\left(k\theta\right)}{\sqrt{\pi}},\quad
					\varphi_{2k}\left(\theta\right)=
					\dfrac{\sin\left(k\theta\right)}{\sqrt{\pi}}
				\end{math}.

			\item[complex]

				\begin{math}
					\phi_{k}\left(\theta\right)=
					\dfrac{e^{ik\theta}}{\sqrt{2\pi}}=
					\dfrac{
						\cos\left(k\theta\right)+i\sin\left(k\theta\right)
					}{\sqrt{2\pi}}
				\end{math}.
		\end{description}
		Then, the Fourier series of $f$ relative of
		${\left\{\varphi_{k}\right\}}_{k\in\mathds{N}}$ and
		${\left\{\phi_{k}\right\}}_{k\in\mathds{N}}$ are
		\begin{columns}
			\begin{column}{0.48\textwidth}
				\begin{description}
					\item[real]

						\begin{math}
							\displaystyle
							\dfrac{a_{0}}{2}+
							\sum\limits_{k\in\mathds{N}}
							a_{k}\cos\left(k\theta\right)+
							b_{k}\sin\left(k\theta\right)
						\end{math}.
				\end{description}
			\end{column}
			\begin{column}{0.48\textwidth}
				\begin{description}
					\item[complex]

						\begin{math}
							\displaystyle
							\sum\limits_{k\in\mathds{Z}}
							\alpha_{k}
							e^{ik\theta},\quad
							\alpha_{k}=
							\frac{1}{2\pi}
							\int\limits_{0}^{2\pi}
							f\left(\theta\right)
							e^{-ik\theta}\dl\theta
						\end{math}.
				\end{description}
			\end{column}
		\end{columns}
	\end{block}
\end{frame}

\begin{frame}
	\frametitle{\secname}
	\begin{block}{Remark~\cite{Herman2021}}
		The subset of functions
		\begin{math}
			{\left\{
				\alert{\frac{1}{\sqrt{2\pi}}},
				\alert{\frac{\cos\left(m\theta\right)}{\sqrt{\pi}}},
				\alert{\frac{\sin\left(n\theta\right)}{\sqrt{\pi}}}
				\right\}}_{m,n\in\mathds{N}}\subset
			L^{2}\left(\left[0,2\pi\right]\right)
		\end{math}
		is an orthonormal subset of
		\begin{math}
			L^{2}\left(\left[0,2\pi\right]\right)
		\end{math}.

		Indeed, $\forall n,m\in\mathds{N}$:
		\begin{itemize}
			\item

			      \begin{math}
				      \displaystyle
				      \int\limits_{0}^{2\pi}
				      {
				      \left(
				      \alert{\frac{1}{\sqrt{2\pi}}}
				      \right)
				      }^{2}
				      \dl\theta=
				      \int\limits_{0}^{2\pi}
				      \frac{1}{2\pi}
				      \dl\theta=
				      \frac{1}{2\pi}
				      {
					      \theta
					      \Bigr|
				      }_{0}^{2\pi}
				      =1
			      \end{math}.

			\item

			      \begin{math}
				      \displaystyle
				      \int\limits_{0}^{2\pi}
				      {
				      \left(
				      \alert{\frac{\cos\left(m\theta\right)}{\sqrt{\pi}}}
				      \right)
				      }^{2}
				      \dl\theta=
				      \int\limits_{0}^{2\pi}
				      \frac{\cos^{2}\left(m\theta\right)}{\pi}
				      \dl\theta=
				      \frac{1}{2\pi}
				      \int\limits_{0}^{2\pi}
				      1+\cos\left(2m\theta\right)
				      \dl\theta=
				      \frac{1}{2\pi}
				      {
					      \left(
					      \theta+
					      \frac{\sin\left(4m\theta\right)}{4m}
					      \right)
					      \Biggr|
				      }_{0}^{2\pi}
				      =1
			      \end{math}.

			\item

			      \begin{math}
				      \displaystyle
				      \int\limits_{0}^{2\pi}
				      {
				      \left(
				      \alert{\frac{\sin\left(n\theta\right)}{\sqrt{\pi}}}
				      \right)
				      }^{2}
				      \dl\theta=
				      \int\limits_{0}^{2\pi}
				      \frac{\sin^{2}\left(n\theta\right)}{\pi}
				      \dl\theta=
				      \frac{1}{2\pi}
				      \int\limits_{0}^{2\pi}
				      1-\cos\left(2m\theta\right)
				      \dl\theta=
				      \frac{1}{2\pi}
				      {
					      \left(
					      \theta-
					      \frac{\sin\left(4m\theta\right)}{4m}
					      \right)
					      \Biggr|
				      }_{0}^{2\pi}
				      =1
			      \end{math}.
		\end{itemize}

		\begin{columns}
			\begin{column}{0.45\textwidth}
				\begin{itemize}
					\item

					      \begin{math}
						      \displaystyle
						      \int\limits_{0}^{2\pi}
						      \alert{\frac{1}{\sqrt{2\pi}}}
						      \alert{\frac{\cos\left(m\theta\right)}{\sqrt{\pi}}}
						      \dl\theta=
						      \frac{1}{\sqrt{2}\pi}
						      \int\limits_{0}^{2\pi}
						      \cos\left(m\theta\right)
						      \dl\theta=
						      0
					      \end{math}.
				\end{itemize}
			\end{column}
			\begin{column}{0.45\textwidth}
				\begin{itemize}
					\item

					      \begin{math}
						      \displaystyle
						      \int\limits_{0}^{2\pi}
						      \alert{\frac{1}{\sqrt{2\pi}}}
						      \alert{\frac{\sin\left(n\theta\right)}{\sqrt{\pi}}}
						      \dl\theta=
						      \frac{1}{\sqrt{2}\pi}
						      \int\limits_{0}^{2\pi}
						      \sin\left(n\theta\right)
						      \dl\theta=
						      0
					      \end{math}.
				\end{itemize}
			\end{column}
		\end{columns}

		\begin{itemize}

			\item

			      \begin{math}
				      \displaystyle
				      \int\limits_{0}^{2\pi}
				      \alert{\frac{\cos\left(m\theta\right)}{\sqrt{\pi}}}
				      \alert{\frac{\sin\left(n\theta\right)}{\sqrt{\pi}}}
				      \dl\theta=
				      \frac{1}{\pi}
				      \int\limits_{0}^{2\pi}
				      \sin\left(n\theta\right)
				      \cos\left(m\theta\right)
				      \dl\theta=
				      \frac{1}{\pi}
				      \int\limits_{0}^{2\pi}
				      \frac{
					      \sin\left(\left(n+m\right)\theta\right)-
					      \sin\left(\left(n-m\right)\theta\right)
				      }{2}
				      \dl\theta=
				      0.
			      \end{math}
		\end{itemize}
	\end{block}
\end{frame}

\begin{frame}
	\frametitle{\secname}

	\begin{definition}[Fourier series generated by $f$]
		Let
		\begin{math}
			f\in L^{2}\left(\left[0,2\pi\right]\right)
		\end{math}.
		The \alert{Fourier coefficients} of $f$ are given by
		\begin{equation*}
			a_{0}=\frac{1}{\pi}
			\int\limits_{0}^{2\pi}
			f\left(\theta\right)\dl\theta,\quad
			a_{k} =
			\frac{1}{\pi}
			\int\limits_{0}^{2\pi}
			f\left(\theta\right)\cos\left(k\theta\right)\dl\theta,\quad
			b_{k} =
			\frac{1}{\pi}
			\int\limits_{0}^{2\pi}
			f\left(\theta\right)\sin\left(k\theta\right)\dl\theta.
		\end{equation*}
		and the \alert{$n$-th partial Fourier sum} is
		\begin{equation*}
			\fcolorbox{DarkBlue}{yellow}{
				\begin{math}
					\displaystyle
					s_{n}f\left(\theta\right)=
					\frac{a_{0}}{2}+
					\sum_{k=1}^{n}
					a_{k}\cos\left(k\theta\right)+
					b_{k}\sin\left(k\theta\right)
				\end{math}.
			}
		\end{equation*}
	\end{definition}

	Indeed, from the equalities $\forall k\in\mathds{N}$:
	\begin{columns}
		\begin{column}{0.25\textwidth}
			\begin{itemize}
				\item

				      \begin{math}
					      \displaystyle
					      \int\limits_{0}^{2\pi}
					      \frac{a_{0}}{2}
					      \dl\theta=
					      \frac{a_{0}}{2}
					      {
						      \theta
						      \Bigr|
					      }_{0}^{2\pi}=
					      \pi a_{0}
				      \end{math}.
			\end{itemize}
		\end{column}
		\begin{column}{0.35\textwidth}
			\begin{itemize}
				\item

				      \begin{math}
					      \displaystyle
					      \int\limits_{0}^{2\pi}
					      \cos\left(k\theta\right)\dl\theta=
					      {
					      \frac{\sin\left(k\theta\right)}{k}
					      \Biggr|
					      }_{0}^{2\pi}=
					      0
				      \end{math}.
			\end{itemize}
		\end{column}
		\begin{column}{0.35\textwidth}
			\begin{itemize}
				\item

				      \begin{math}
					      \displaystyle
					      \int\limits_{0}^{2\pi}
					      \sin\left(k\theta\right)\dl\theta=
					      {
					      \frac{-\cos\left(k\theta\right)}{k}
					      \Biggr|
					      }_{0}^{2\pi}=
					      0
				      \end{math}.
			\end{itemize}
		\end{column}
	\end{columns}
\end{frame}

\begin{frame}
	If we integrate the Fourier series term by term
	\begin{align*}
		\alert{\int\limits_{0}^{2\pi}}
		f\left(\theta\right)
		\dl\theta & =
		\alert{\int\limits_{0}^{2\pi}}
		\frac{a_{0}}{2}
		\dl\theta+
		\alert{\int\limits_{0}^{2\pi}}
		\left(
		\sum_{k=1}^{\infty}
		a_{k}\cos\left(k\theta\right)+
		b_{k}\sin\left(k\theta\right)
		\right)
		\dl\theta.
		\shortintertext{Then,}
		\int\limits_{0}^{2\pi}
		f\left(\theta\right)
		\dl\theta & =
		\frac{a_{0}}{2}
		\int\limits_{0}^{2\pi}
		\dl\theta+
		\sum_{k=1}^{\infty}
		\left(
		a_{k}
		\alert{
			\int\limits_{0}^{2\pi}
			\cos\left(k\theta\right)
			\dl\theta
		}
		+
		b_{k}
		\alert{
			\int\limits_{0}^{2\pi}
			\sin\left(k\theta\right)
			\dl\theta
		}
		\right).      \\
		\int\limits_{0}^{2\pi}
		f\left(\theta\right)
		\dl\theta & =
		\pi a_{0}+
		\sum_{k=1}^{\infty}
		\left(
		a_{k}\cdot\alert{0}+
		b_{k}\cdot\alert{0}
		\right).\quad\implies
		\boxed{\color{DarkBlue}
			a_{0}=
			\frac{1}{\pi}
			\int\limits_{0}^{2\pi}
			f\left(\theta\right)\dl\theta.
		}
	\end{align*}
\end{frame}

\begin{frame}
	Multiplying the Fourier series by
	\alert{$\cos\left(m\theta\right)$},
	$m\in\mathds{N}$
	and integrating term by term:
	\begin{align*}
		\int\limits_{0}^{2\pi}
		\alert{\cos\left(m\theta\right)}
		f\left(\theta\right)
		\dl\theta
		 & =
		\int\limits_{0}^{2\pi}
		\alert{\cos\left(m\theta\right)}
		\frac{a_{0}}{2}
		\dl\theta+
		\int\limits_{0}^{2\pi}
		\alert{\cos\left(m\theta\right)}
		\left(
		\sum_{k=1}^{\infty}
		a_{k}\cos\left(k\theta\right)+
		b_{k}\sin\left(k\theta\right)
		\right)
		\dl\theta. \\
		\int\limits_{0}^{2\pi}
		f\left(\theta\right)
		\cos\left(m\theta\right)
		\dl\theta
		 & =
		0+
		\sum_{k=1}^{\infty}
		\left(
		a_{k}
		\int\limits_{0}^{2\pi}
		\cos\left(k\theta\right)
		\cos\left(m\theta\right)
		\dl\theta+
		b_{k}
		\int\limits_{0}^{2\pi}
		\sin\left(k\theta\right)
		\cos\left(m\theta\right)
		\dl\theta
		\right).   \\
		\int\limits_{0}^{2\pi}
		f\left(\theta\right)
		\cos\left(m\theta\right)
		\dl\theta
		 & =
		\sum_{k=1}^{\infty}
		\left(
		\frac{a_{k}}{2}
		\int\limits_{0}^{2\pi}
		\alert{
			\cos\left(\left(m+k\right)\theta\right)+
			\cos\left(\left(m-k\right)\theta\right)
		}
		\dl\theta+
		\frac{b_{k}}{2}
		\int\limits_{0}^{2\pi}
		\alert{
			\sin\left(\left(m+k\right)\theta\right)+
			\sin\left(\left(m-k\right)\theta\right)
		}
		\dl\theta
		\right).
		\shortintertext{
			When $m\neq k$ \alert{both integrals} vanish, thus the
			infinite sum reduces to $m$-th addend.}
		\int\limits_{0}^{2\pi}
		f\left(\theta\right)
		\cos\left(m\theta\right)
		\dl\theta
		 & =
		a_{m}
		\int\limits_{0}^{2\pi}
		\alert{\cos^{2}\left(m\theta\right)}
		\dl\theta+
		b_{m}
		\alert{
			\int\limits_{0}^{2\pi}
			\sin\left(m\theta\right)
			\cos\left(m\theta\right)
			\dl\theta
		}.         \\
		\int\limits_{0}^{2\pi}
		f\left(\theta\right)
		\cos\left(m\theta\right)
		\dl\theta
		 & =
		\frac{a_{m}}{\alert{2}}
		\int\limits_{0}^{2\pi}
		\alert{1+\cos\left(2m\theta\right)}
		\dl\theta
		+
		b_{m}\cdot
		\alert{0}. \\
		\int\limits_{0}^{2\pi}
		f\left(\theta\right)
		\cos\left(m\theta\right)
		\dl\theta
		 & =
		a_{m}\pi.
		\quad
		\implies
		\boxed{\color{DarkBlue}
			a_{m}=
			\frac{1}{\pi}
			\int\limits_{0}^{2\pi}
			f\left(\theta\right)
			\cos\left(m\theta\right)
			\dl\theta.
		}
	\end{align*}
\end{frame}

\begin{frame}
	Multiplying the Fourier series by
	\alert{$\sin\left(m\theta\right)$},
	$m\in\mathds{N}$
	and integrating term by term:
	\begin{align*}
		\int\limits_{0}^{2\pi}
		\alert{\sin\left(m\theta\right)}
		f\left(\theta\right)
		\dl\theta
		 & =
		\int\limits_{0}^{2\pi}
		\alert{\sin\left(m\theta\right)}
		\frac{a_{0}}{2}
		\dl\theta+
		\int\limits_{0}^{2\pi}
		\alert{\sin\left(m\theta\right)}
		\left(
		\sum_{k=1}^{\infty}
		a_{k}\cos\left(k\theta\right)+
		b_{k}\sin\left(k\theta\right)
		\right)
		\dl\theta. \\
		\int\limits_{0}^{2\pi}
		f\left(\theta\right)
		\sin\left(m\theta\right)
		\dl\theta
		 & =
		0+
		\sum_{k=1}^{\infty}
		\left(
		a_{k}
		\int\limits_{0}^{2\pi}
		\cos\left(k\theta\right)
		\sin\left(m\theta\right)
		\dl\theta+
		b_{k}
		\int\limits_{0}^{2\pi}
		\sin\left(k\theta\right)
		\sin\left(m\theta\right)
		\dl\theta
		\right).   \\
		\int\limits_{0}^{2\pi}
		f\left(\theta\right)
		\sin\left(m\theta\right)
		\dl\theta
		 & =
		\sum_{k=1}^{\infty}
		\left(
		\frac{a_{k}}{2}
		\int\limits_{0}^{2\pi}
		\alert{
			\sin\left(\left(m+k\right)\theta\right)+
			\sin\left(\left(m-k\right)\theta\right)
		}
		\dl\theta+
		\frac{b_{k}}{2}
		\int\limits_{0}^{2\pi}
		\alert{
			\cos\left(\left(m-k\right)\theta\right)-
			\cos\left(\left(m+k\right)\theta\right)
		}
		\dl\theta
		\right).
		\shortintertext{
			When $m\neq k$ \alert{both integrals} vanish, thus the
			infinite sum reduces to $m$-th addend.}
		\int\limits_{0}^{2\pi}
		f\left(\theta\right)
		\sin\left(m\theta\right)
		\dl\theta
		 & =
		a_{m}
		\alert{
			\int\limits_{0}^{2\pi}
			\cos\left(m\theta\right)
			\sin\left(m\theta\right)
			\dl\theta
		}+
		b_{m}
		\int\limits_{0}^{2\pi}
		\alert{\sin^{2}\left(m\theta\right)}
		\dl\theta. \\
		\int\limits_{0}^{2\pi}
		f\left(\theta\right)
		\sin\left(m\theta\right)
		\dl\theta
		 & =
		a_{m}\cdot
		\alert{0}+
		\frac{b_{m}}{\alert{2}}
		\int\limits_{0}^{2\pi}
		\alert{1-\cos\left(2m\theta\right)}
		\dl\theta. \\
		\int\limits_{0}^{2\pi}
		f\left(\theta\right)
		\sin\left(m\theta\right)
		\dl\theta
		 & =
		b_{m}\pi.
		\quad
		\implies
		\boxed{\color{DarkBlue}
			b_{m}=
			\frac{1}{\pi}
			\int\limits_{0}^{2\pi}
			f\left(\theta\right)
			\sin\left(m\theta\right)
			\dl\theta.
		}
	\end{align*}
\end{frame}